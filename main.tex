\documentclass{article}
\usepackage[utf8]{inputenc}
\usepackage[spanish]{babel}
\usepackage{listings}
\usepackage{graphicx}
\graphicspath{ {images/} }
\usepackage{cite}

\begin{document}

\begin{titlepage}
    \begin{center}
        \vspace*{1cm}
            
        \Huge
        \textbf{Parcial I- Calistenia}
            
        \vspace{0.5cm}
        \LARGE
        Subtítulo
            
        \vspace{1.5cm}
            
        \textbf{Julián Pérez López}
            
        \vfill
            
        \vspace{0.8cm}
            
        \LARGE
        Departamento de Ingeniería Electrónica y Telecomunicaciones\\
        Universidad de Antioquia\\
        Medellín\\
        Marzo de 2021
            
    \end{center}
\end{titlepage}

\tableofcontents
\newpage
\section{Introducción}\label{intro}
Orientar mediante diversas instrucciones, el procedimiento necesario para completar una acción en la que se necesitarán diversos objetos y cierto grado básico de concentración.
\section{Materiales Necesarios} \label{contenido}
Lo que necesitaremos para este proyecto es: 
\subsection{2 targetas de igual tamaño, material y peso}
\subsection{Una hoja de papel}

\section{Pasos a seguir} 
Para este ejercicio solo podrá disponer de una mano de lo contrario no estaría realizando el ejercicio de manera correcta, tambien recuerde que no puede utilizar ningún otro elemento que lo ayude a sostener ni la hoja de papel, ni las dos tarjetas ya que igual que lo anterior estaría mal realizado \\


En una superficie plana y sobre la hoja de papel, tomar las dos tarjetas del mismo tamaño con su mano más hábil, posterior a esto intente ubicarlas de forma vertical la una pegada a la otra, luego ubique el dedo índice en la parte superior de estas, de tal manera que sostenga las dos tarjetas con este dedo, luego con su dedo anular y el pulgar sujete una de las tarjetas en cada uno de los lados (sin soltar el dedo índice de la parte superior), y comience a desplazar una de las tarjetas hacía atrás hasta conseguir la forma piramidal requerida.\\

Al realizar estos pasos, verifique que la figura piramidal se sostenga sin ninguna clase de ayuda incluyendo sin sus manos por lo menos por un tiempo estimado de 10 segundos.
\bibliographystyle{IEEEtran}


\end{document}
